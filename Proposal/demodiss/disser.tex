\documentclass[12pt,twoside,notitlepage]{report}
\usepackage{a4}
\usepackage{verbatim}
\usepackage{graphicx}
\usepackage[font={small,it}]{caption}
\usepackage{pdfpages}
\raggedbottom                           % try to avoid widows and orphans
\sloppy
\clubpenalty1000%
\widowpenalty1000%

\addtolength{\oddsidemargin}{6mm}       % adjust margins
\addtolength{\evensidemargin}{-8mm}

\renewcommand{\baselinestretch}{1.1}    % adjust line spacing to make

\begin{document}

\bibliographystyle{plain}


%%%%%%%%%%%%%%%%%%%%%%%%%%%%%%%%%%%%%%%%%%%%%%%%%%%%%%%%%%%%%%%%%%%%%%%%
% Title


\pagestyle{empty}

\hfill{\LARGE \bf Karina Palyutina}

\vspace*{60mm}
\begin{center}
\Huge
{\bf Machine learning inference of search engine heuristics} \\
\vspace*{5mm}
Part II Project \\
\vspace*{5mm}
St Catharine's College \\
\vspace*{5mm}
\today  % today's date
\end{center}

\cleardoublepage

%%%%%%%%%%%%%%%%%%%%%%%%%%%%%%%%%%%%%%%%%%%%%%%%%%%%%%%%%%%%%%%%%%%%%%%%%%%%%%
% Proforma, table of contents and list of figures

\setcounter{page}{1}
\pagenumbering{roman}
\pagestyle{plain}

\chapter*{Proforma}

{\large
\begin{tabular}{ll}
Name:               & \bf Karina Palyutina                       \\
College:            & \bf St Catharine's College                     \\
Project Title:      & \bf Machine learning inference of search engine heuristics \\
Examination:        & \bf Part II Project        \\
Word Count:         & \bf \footnotemark[1]     \\
Project Originator: & Dr Jon Crowcroft                    \\
Supervisor:         & Dr Jon Crowcroft                  \\ 
\end{tabular}
}
\footnotetext[1]{This word count was computed
by {\tt detex diss.tex | tr -cd '0-9A-Za-z $\tt\backslash$n' | wc -w}
}
\stepcounter{footnote}

\newpage


\section*{Original Aims of the Project}


\section*{Work Completed}


\section*{Special Difficulties}

\section*{Declaration of Originality}

I, Karina Palyutina of St Catharine's College, being a candidate for Part II of the Computer
Science Tripos , hereby declare
that this dissertation and the work described in it are my own work,
unaided except as may be specified below, and that the dissertation
does not contain material that has already been used to any substantial
extent for a comparable purpose.

\bigskip
\leftline{Signed }

\medskip
\leftline{Date }

\cleardoublepage

\tableofcontents


%%%%%%%%%%%%%%%%%%%%%%%%%%%%%%%%%%%%%%%%%%%%%%%%%%%%%%%%%%%%%%%%%%%%%%%
% now for the chapters

\cleardoublepage        % just to make sure before the page numbering
                        % is changed

\setcounter{page}{1}
\pagenumbering{arabic}
\pagestyle{headings}

\chapter*{Introduction}
\addcontentsline{toc}{chapter}{Introduction}
This project is inspired by increasing importance of search engine rankings.
Today major search engines given a query return web pages in an order
determined by secret algorithms. Such algorithms are believed 
to incorporate multiple unknown factors.
For instance, Google claims to have over 200 unique factors that influence a
position of a webpage in the search results relative to a query
\footnote{http://www.google.com/competition/howgooglesearchworks.html}. Only
a handful of these factors are disclosed to the webmasters  in the form of very
general guidelines. Moreover, the Google algorithm in particular is updated
frequently. However, most of the knowledge around the area amounts to
speculation. Despite the fact that it is possible to pass a vast number of
queries through the black box of any existing search engine,the immensity of
the search space, and instability of such algorithms make them impossible to
reverse engineer.

Machine learning is a natural approach to inferring the true algorithm from a
subset of all possible observations. However, applying machine learning
techniques to real search engines would be hardly effective, as the dynamic
nature of the algorithms and the web as well as lack of meaningful feedback
would prevent incremental improvement: when there are as many as 200 features
in question, false assumptions made by a learner may have an unpredictable
effect on its performace.

More generally, there are certain ambiguities associated with machine learning,
which are 'problem-specific'. For example, it proves difficult to decide how
much training data is necessary, as well as and selecting it to avoid
over/under-fitting\cite{domingos}. Similarly, it is not straightforward which
machine learning technique is best for a particular problem.

This project is concerned with application of machine learning techniques to
search engines. The aim of the project, in particular,  is to explore how
machine learning techniques can be used effectively to infer algorithms from
search engines. To address the limitations imposed by existing search engines,
part of the task is to develop a toy search engine that allows me to comtrol
the nature and complexity of used heuristics. Such transparency addresses the
problems stated above and, more importantly,  allows for useful evaluation of
machine learning techniques by providing meaningful feedback.

Even though this study does not attempt to reverse engineer any existing
heuristics, the results can be applied to such an ambitious task.
Moreover, such a framework is potentially more general and can be used for a
range of problems.

\newcommand\todo[1]{\textcolor{red}{#1}}
\todo{overview of the chapters here.}

\cleardoublepage
\chapter{Preparation}

This chapter describes work that has been done before coding was started.  In
particular, it focuses on the reasoning behind the design of the system to be
implemented. The first section is devoted to research undertaken to determine
what can be done and how best to do it. The second section formulates the
system requirements, namely formalizes everything that is developed in this
project. The last section outlines the particulars of the software engineering
approach to be adopted by this project.

\section{Formulating the Goals}
A particular difficulty in this project has been in planning what has to be
done. Due to the exploratory nature of the project the course of action had to
be predominantly determined by the outcome of a current tactic. Moreover, the
unknowns originating from the machine learning further complicated matters. 

\paragraph{System Overview.}

To achieve the goal of the project, a machine learning techniques comparison
framework was necessary. In the Introduction I mentioned the benefit of having
a transparent system as an object of learning. To further justify this
decision, it is worth mentioning that  generalisation using machine learning is
different from most optimization problems in that the function that is being
optimized is out of our reach, and all that is visible to the machine learner
is the training error. Because our goal is not the correct classification of
real data, but identifying the means to correct classification, it is important
that informed choices are made towards improvement of the learner. Taking this
into account, knowing the function that we want to learn and having direct
control over it  will guide the improvement of the search engine. 

This argument motivates a system in three parts: a search engine, a machine
learner and a parser to mediate between the two.
Figure \ref{overview} illustrates the proposed learning system. Training data
is a set of web pages set aside specifically for training purposes. Such web
pages are not required to have any special properties, but diversity and
typicality are seen as advantageous. As for the size of the training data, 
Domingos \cite{domingos} suggests that a primitive learner given more data
performs better than a more complex one with less data. This, of course, is
under certain assumptions of data quality, namely the assumption that the
training data is a representative subset of all the possible data. Intuitively,
provided there is no bias in data gathering, more data implies better
generality. I have started with a training set spanning an order of a few
thousands of pages, however, in practice, I found that there is no
particular improvement beyond a thousand pages. \todo{Link to relevant part or
example data here?}

\paragraph{Data.}

\paragraph{Search Engine.}

Next important decision regarded the search engine.  Originally, I considered
using open source existing engines, in particular, Lucene. Even though I could
freely modify it for the purposes of the project, the complexity of it was
superfulous. I saw writing a simple search engine as a more beneficial
exercise, as developing it in the first place potentially gives an insight into
the problem.

Functionally our search engine is a black box that takes a set of webpages and
a set of queries and outputs an order. Or a score!

\paragraph{Language.}

When choosing a programming language, main considerations reduced to library
availibility and simplicity. The project imposes no special  requirements on the
language, apart from, perhaps, library infrastructure for parsing web pages.
Python is simple language with extensive library support. As
for efficiency, all the mathematical operations in this project rely on python
math libraris, which are implemented in C. I have not programmed in Python
before the project, so a slight overhead was caused by having to learn a new
language.

\paragraph{Machine Learning.}

I have now covered main peripheral decisions, but it is machine learning that
constitutes the central part of the project. The field was completely new to me
to start with, so research of different techniques was a big part of the
preparation.

It is generally recommended that the simplest learners are tried
first\cite{domingos}. Of all learners Naive Bayesian is one of the most
comprehensible.This in itself is a major advantage according to the Occam's
razor principle, which finds ample application in machine learning.

Naive Bayes is a probabilistic classifier based on the Bayes Theorem. The
posterior probability \(P(C|\vec{F})\) denotes the probability that a sample
page with a feature vector \(\vec{F}=(F_1,F_2,\dots,F_n)\) belongs to class C.
The posteriior probability is computed from the observable in the training data: the prior
probability \(P(C)\) -- the unconditional probability of a page belonging to
the class C, the likelihood \(P(\vec{F}|C)\) and the evidence \(P(\vec{F})\):
\begin{equation}
P(C|\vec{F}) = \frac{P(C)P(\vec{F}|C)}{P(\vec{F})}
\end{equation}

The simplicity of Bayesian approach owes to the conditional independence
assumption: each \(F_i\) in \(\vec{F}\) is assumed to be independent of one
another to get \(P(\vec{F}|C)=P(F_1|C)*P(F_2|C)*\dots*P(F_n|C)\). This leads to a concise classifier definition:
\begin{equation}
\hat{C}= argmax_C P(C)\prod_{i=1}^{n}P(F_i|C)
\end{equation}
where \(C\) is the result of classification of a page with feature vector
\(F_1,F_2,\dots,F_n\).

In practice, the crude assumption rarely  holds and is likely to
be violated by our data, as we expect features of pages to be interdependent.
However, it has been shown that Naive Bayes performs well under zero-one loss
function in presence of dependencies\cite{domingos96}. This has a few
implications for this project, particularly, on evaluation methods. 


\paragraph{Evaluation methodology.}
Baseline:Bayes and ceiling
\section{Requirements Analysis}
Search engine: efficiency!
Machine learner: simplicity (domingos) occam's razor
Interface between: hiding, encapsulating
Evaluability
\begin{figure}
\centering
\includegraphics[scale=0.5]{figs/overview.pdf}
\caption{Overview of the system. Three major parts from left to right are search engine,
parser and machine learner.}
\label{overview}
\end{figure}

Figure \ref{eval} shows the evaluation strategy. The Test Data is the data
carefully set aside at the beginning that is never exposed to the learner. 

Ranking vs scoring 
\begin{figure}
\centering
\includegraphics[scale=0.5]{figs/eval.pdf}
\caption{Evaluation system. }
\label{eval}
\end{figure}
\section{Development Strategy}

\cleardoublepage
\chapter{Implementation}
This section describes parts of the system that I have implemented.

\section{Search Engine}
Overview

\subsection{PageRank}
\subsection{Indexer}
\subsection{Optimization}

\section{Parser}

\section{Machine Learning}
Overview
\subsection{Naive Bayes}

\subsection{Support Vector Machine}

\cleardoublepage
\chapter{Evaluation}

\cleardoublepage
\chapter{Conclusion}
\cleardoublepage

%%%%%%%%%%%%%%%%%%%%%%%%%%%%%%%%%%%%%%%%%%%%%%%%%%%%%%%%%%%%%%%%%%%%%
% the bibliography

\addcontentsline{toc}{chapter}{Bibliography}
\bibliography{refs}
\cleardoublepage

%%%%%%%%%%%%%%%%%%%%%%%%%%%%%%%%%%%%%%%%%%%%%%%%%%%%%%%%%%%%%%%%%%%%%
% the appendices
\appendix

\chapter{Project Proposal}

%

\title{Machine learning inference of search engine heuristics}
\subtitle{Part II Computer Science Project Proposal}
\author{K. Palyutina, St. Catharine's College \\
        Originator: Dr. Jon Crowcroft}


% Main document

\section*{\bf Introduction, The Problem To Be Addressed}
PageRank (an algorithm which is used by Google to evaluate the `importance' of a web page) is one of the most crucial factors which determine page performance in search returns. However, there are many more of such factors that are believed to become increasingly influential. Because Google's algorithm is frequently revised, changing page ranks cause web site owners to speculate about how their web pages `deserved' an upgrade or a downgrade. Despite a large interest in this area, little research has been done to determine to what degree such factors affect the performance of a page. Certain tools\footnote{For example, Woorank or SEO are the most popular Chrome extensions to assess certain page qualities.} exist which attempt to advise web masters how to `improve' their pages. However, heuristics used by such tools are not known and are possibly incomplete and no attempt has come close to accurately predicting Google rankings.

A problem of approximating algorithms which may be used by modern search engines is characterised by vast search space, which makes exhaustive search impossible and introduces the need for generalisation. Such a problem can be reduced to a classification problem, which is traditionally solved with the help of machine learning techniques. Even though machine learning finds natural application in this area, it is easy to see how it would be very hard to create a learner and apply it to, for example, Google's search engine. Naturally, one would need to have exhaustive resources to conduct such a study. Besides, machine learning has major drawbacks that would hinder such an ambitious experiment.

Firstly, there are little theoretical guarantees in this approach. Bounds, if any, referring to how much data needs to be processed to produce a `correct' classifier are very imprecise and a classifier that performs well in practice may not be `true'\footnote{A classifier is `true' if it classifies data correctly for all inputs.}. This means that if a machine learner was to be trained by the `real world' data (Google search returns), little could be said about the performance of the obtained algorithm or, indeed, the `truthfulness' of it. Not only would it give little insight into how successful the learning is, but also no guidance for improvement. 

Another similar issue is referred to as `overfitting': this describes a situation in which a classifier performs outstandingly on a particular set of data (often similar to training data), but given different data will perform as badly as random selection. This occurs when false connections between features and outputs have been made by the learner. Unfortunately, there is no single technique that will always avoid over/under-fitting\cite{domingos}.

Clearly, such limitations are hard to combat. Besides, machine learning techniques vary greatly, so clear and detailed feedback is essential to draw conclusions about the performance of a learner. Knowing how a particular technique copes with certain heuristics would be valuable, as it would allow to approach the original problem (approximating search engine heuristics reliably) in an informed way.  

In this light, this project aspires to explore how machine learning techniques can be used to infer algorithms from search engines. To battle the constraints described above I will write my own simple search engine which will be used to observe the effectiveness of different machine learning techniques (see Figure \ref{diag1}). The existence of such a search engine is vital to the project, as it offers ultimate control over the learning process. The heuristics used in the search engine will be transparent, which, for example, eliminates the dependency on continuously changing heuristics used by Google. Another advantage of this approach is the fact that only a minimal fraction of the web needs to be used. Such a `mini-internet' will reside offline, which will speed up the indexing and also allow local modification of web pages. 

\begin{figure}
\centering
\includegraphics[scale=0.5]{diagram1}
\caption{The training of the learner. The system enclosed within the dotted box will be implemented in this project. The mini web is created by cloning web pages from the internet. The learner queries the search engine and gets back the results of the query as an input. }
\label{diag1}
\end{figure}

Most importantly, this approach gives me straightforward ways to reason about the performance of learning techniques. The search engine can be evolved to incorporate various heuristics together with PageRank. Such heuristics need not match Google's actual heuristics, however must aim to improve browsing experience \footnote{`Precision and Recall' method can be used as a guide to evaluation of search engine complexity.}. A lot of speculation has been done by web masters as to which qualities of a web page affect its ranking, these include compliance with web standards, number of words per page, frequency of occurrence of the search term on the page and alike. Because the goal of the project does not include `reverse engineering' any particular existing algorithm, there is no harm in using such guesses as guidance. 

Evolving the search engine and, hence, the learner iteratively will result in comprehensive conclusions about the effectiveness of the machine learning technique in question. Such conclusions can be used in the future as a guidance to learner design. 

\section*{\bf Starting Point}

\begin{itemize}
\item A project\footnote{\url{http://www.scienceforsearch.com/project1.asp}} was undertaken by the proposer, which developed a primitive algorithm to predict, given six characteristics of a web page, its Google ranking.  This project is mainly an inspiration, however, the speculations about the Google page ranking factors can be useful for the search engine design. 
\item Python packages exist for manipulating web pages.
\item Wget is a Linux open source utility that can be used to clone web pages.
\item The paper describing PageRank is published and will be used to implement the algorithm.
\end{itemize}


\section*{\bf Resources Required}
\begin{itemize}
\item For this project I shall mainly use my own dual-core computer that runs Ubuntu Linux. I accept full responsibility for this machine and I have made contingency plans to protect myself against hardware and/or software failure.
\item Backup will be to a BitBucket repository and/or an external hard drive.
\item I will work on MCS computers should my main machine suddenly fail. 
\end{itemize}
\section*{\bf Work to be done}

The project breaks down into the following sub-projects:

\begin{enumerate}
\item Decide on a category of search terms to explore in order to create a small network consisting of relevant web pages.

\item Implement PageRank within this network. 

\item Write a simple search engine incorporating PageRank and few other features. 

\item Decide on the representation of the input for the learner and set up the framework to format it. 

\item In advance set aside training and test data: this is necessary to then justify the evaluation of the classifier.

\item Write a simple prototype for the learner\footnote {A Naive Bayesian would be a good prototype to use.} to test the grounds. Evaluate its performance to then set goals for the final learner.  

\item Design, implement and test the learner. 

\item Attempt to evolve the search engine to be more usable and complex and observe how the learner copes with the changes of the search engine. 


\end{enumerate}

\section*{\bf Success Criterion for the Main Result}


The project will be a success if... 
\begin{itemize}
\item The resulting classifier can identify the importance of the PageRank factor in the given search engine.
\item The results of the experiment show how the chosen machine learning technique deals with various search engine heuristics. I would especially like to observe that certain heuristics are harder to pick up on than others and vice versa.
\end{itemize}
\section*{\bf Possible Extensions}
If I achieve my main result early I shall experiment with other machine learning techniques to see which perform better. I could also apply my learner to real search engines such as Google and Bing in the hope of 
discovering dependencies between features of the page and its success in ranking results.
\section*{\bf Timetable: Work plan and Milestones to be achieved.}


Planned starting date is 19/10/2011.

\begin{enumerate}

\item {\bf 9 Oct - 19 Oct:} 
\begin{itemize}
\item Do preliminary reading.
\item Familiarize myself with the field of machine learning.
\end{itemize} 
{\bf Milestone: } Complete project proposal. 
\item {\bf Oct 20 - Nov 3:} 
    \begin{itemize}
    \item Decide which and how many websites should be cloned for use as the mini web.  
    \item Prepare some training data and, separately, test data. This includes queries to be run on the search engine and expected results. 
    \end{itemize}
\item {\bf Nov 4 - Nov 15:} 
    \begin{itemize}
    \item Start writing a simple search engine and evaluate it on the test data. 
    \end{itemize}
\item {\bf Nov 15 - Nov 25:} 
    \begin{itemize}
    \item Finish the search engine.
    \item Start developing an early prototype for the learner. 
    \end{itemize} 
    {\bf Milestone: } Have a prototype of a complete system.
\item {\bf Nov 25 - Dec 15:} 
\begin{itemize}
\item Evaluate the performance of the prototype learner. 
\item Design and start implementing the final learner using the results obtained from the prototype as guidance. 
\end{itemize}
\item {\bf Dec 16 - Jan 1:} Finish the implementation of the learner. 
\item {\bf Jan 2 - Jan 16:} 
     Evaluate the resulting classifier. Here is also good time to try a different design for the learner if the classifier does not perform as well as intended.
\item {\bf Jan 17 - Feb 1:} Start working on progress report.  
{Milestone: } Write progress report. 
\item {\bf Feb 2 - Feb 20} Implement extensions.

\item {\bf Feb 20 - Mar 5:} Evaluate extensions. 

\item {\bf Mar 5 - Mar 25:} Write dissertation main chapters.

\item {\bf Mar 25 - April 10}  Further evaluation and complete dissertation.
{Milestone: } Dissertation final draft is finished.

\item {\bf April 11 - April 20:} Proof reading and submission.

\end{enumerate}


 



\end{document}
